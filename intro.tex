Writing a research paper or a term paper needs a few things to think about. 

We will use bullet point and numbered lists to state our points:
The following is a numbered list:

\begin{enumerate}
    \item First, you will need to find a topic to write.
    \item Second, you will need a structure
    \item Third, you will write an outline
    \item Fourth, you will need plenty of time to revise and edit
\end{enumerate}

The following is an example of a bullet-point list for developing a critical appraisal of a topic

\begin{itemize}
    \item We will need to identify question and express it in PICO format
    \item We will then search scholarly databases with key search terms
    \item We will then identify as many relevant papers as we can based on our inclusion and exclusion criteria
    \item We will use the GRADE criteria to assess the body of evidence based on the outcomes we are interested in
    \item We will synthesise the findings of the studies using principles of meta-analysis to arrive at a summary estimate
\end{itemize}

What has happened? In each case, as we decided that we want a list (enumerated list or itemised list), we started and ended with a `begin' and `end' words with the identity of the list in curly brackets. This `identity of the list' is referred to as an `environment'. Here is a list of latex environments:

\url{https://latex.wikia.org/wiki/List_of_LaTeX_environments}

The above was an example of adding an URL that became a clickable link when you view it in the PDF. This is an example of a `command' within the document that you would use. The command in this case is `\\url'. Note that in order to use this command, we need to include in our `main.tex' another command called \\usepackage{hyperref}; if we did not do that, then we would not be able to use url command or the url command would throw an error. Therefore, keep these in mind:

\begin{itemize}
    \item Write in plain text
    \item Identify what packages you will use in your paper
    \item List them in the main.tex document
    \item Figure out what environments you will use in your paper
\end{itemize}{}